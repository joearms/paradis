\documentclass[12pt]{hitec}

\usepackage{fancyvrb}
%%\usepackage{hyperref}

\title{Problem set 1}
\author{Joe Armstrong}

\begin{document}

\maketitle

\tableofcontents

\section{Introduction}

These problems are for lectures E1 and E2.

Additional material can be found in Chapters 1..6 of
the course book {\sl Programming Erlang, 2'nd edition}. 

\section{How to test your code}

We will use TDD\footnote{Test Driven Development.} to develop the code
and to decide when to stop working on the code.
Here's what you have to do:

\begin{enumerate}
\item Make sure Erlang runs on your machine.

\item Download the file \verb+problems1.erl+ from my github repository
\newline
  \verb+http://joearms.github.com/paradis1/src/problems1.erl+.

\item Implement the module \verb+solutions1.erl+ (and any modules
that \verb+solutions1+ calls).

\item Compile \verb+problems1.erl+, \verb+solutions1.erl+ and
any other code you need.

\item Evaluate the function \verb+problems1:test_easy()+. If this
  evaluates to \verb+horray+ your program has passed
  its unit test and will be graded as {\sl Godk\"{a}nd}.

\item To get {\sl V\"{a}l Godk\"{a}nd} you should also pass the tests
  in the function \verb+problems1:test_hard()+.

\end{enumerate}

\section{Problems}

All your code should be in the module \verb+solutions.erl+ it
should export the functions \verb+factorial/1+, \verb+rotate/2+ etc.
(The notation \verb+F/N+ means the function \verb+F+ with \verb+N+
arguments).  The functions that you have to implement are those which
are called from \verb+problems1.erl+. Each function is described
in a separate subsection below:

\subsection{factorial(N)}

Compute \verb+factorial(N)+ where \verb+N+ is an integer.

Factorial \verb+N+ is defined as \verb+N * N-1 * N-2 * ... * 1+

For example \verb+factorial(5) = 5 * 4 * 3 * 2 * 1+ which is
\verb+120+

Note: As you can see, the test function in 
\verb+solutions.erl+\footnote{The listing is at the end of this document.} 
starts:

\begin{Verbatim}[frame=single]
test() ->
    120 = solutions:factorial(5),
    ...
\end{Verbatim}

This is a unit test that tests that you can correctly compute
\verb+factorial(5)+.

\subsection{rotate(N, L)}

\verb+rotate(1, L)+ rotates the list \verb+N+ by one place.
For example a one place rotation of the list \verb+[a,b,c,d,e,f]+ is the list
\verb+[b,c,d,e,f,a]+. The element at the head of the list \verb+a+ is moved
to the end of the list.

\verb+rotate(N, L)+ for positive \verb+N+ performs a one place rotation \verb+N+ times.

\verb+rotate(-1,L)+ rotates the list in the opposite direction and moves the
last element of the list to the beginning 
so  \verb+rotate(-1, [a,b,c,d,e,f])+ is the list
\verb+[f,a,b,c,d,e]+.

\verb+rotate(N, L)+ for negative \verb+N+ performs \verb+rotate(-1, L)+ \verb+N+ times.

\subsection{count\_atoms(A, X)}

Returns the number of Erlang atoms \verb+X+ in the data structure \verb+X+.

In the test case we parse the program \verb+problems1.erl+.  The
module \verb+epp+ exports \verb+parse_file+ which returns the parse
tree of the program. The parse tree is dumped into the file
\verb+debug+ so you can see what it looks like. The first time you run
the program you should take a look in the file and see what the parse
tree of the program looks like.

Having dumped the parse tree, take a look at the parse tree and see if
you can figure out what the internal representation of an atom
is. Then write the function \verb+count(A, X)+ to count the number of
occurrences of the atom \verb+A+ in the data structure \verb+X+.

{\sl Hint: Use the BIF\footnote{Built in function} \verb+tuple_to_list+}

\subsection{expand\_markup(Str)}

\verb+expand_markup(Str)+ expands a simple markup language into HTML.
``\verb+**+'' is a bold toggle. The first occurrence of \verb+**+ in the string
is replaced by \verb+<b>+, the second by \verb+<\b>+ and so on.

Thus \verb+expand_markup("**BB**")+ expands to \verb+<b>BB</b>+.

``\verb+__+'' toggles an italic flag. So \verb+expand_markup("__II__")+ evaluates to
\verb+<i>II</i>+.

HTML fragments should be ``well nested'' so
\verb+<b>..<i>..</i>..</b>+ is well nested, but \verb+<b>..<i>..</b>..</i>+ is
badly nested.

\verb+expand_markup+ should always return well nested HTML. 
So, for example, \verb+expand_markup("**B__I**C__")+ should expand
to\\
\verb+<b>B<i>I</i></b><i>C</i>+.

Note: Designing markup languages might look simple, but they have
subtleties that are often missed. The last example is an example of
this.

Getting \verb+expand_markup+ right for all inputs is difficult.  Don't
worry if this exercise takes you a few hours to solve.

If you fail to solve the tricky case comment out the test in the unit
test program and do the other exercises.

\section{A successful testrun}

Here's what it looks like if everything goes right:

\begin{Verbatim}[frame=single]
$ erl
Erlang R16B (erts-5.10.1) ...
Eshell V5.10.1  (abort with ^G) ...
1> c(problems1).
{ok,problems1}
2> c(solutions).
{ok,solutions}
3> problems1:test_east().
horray
4> problems1:test_east().
written:"debug"
whoopy
\end{Verbatim}

\section{Hints}

\begin{itemize}

\item Print small data structures in the shell with \verb+io:format("~p", [X])+.

\item Use \verb+problems1:dump/2+ to pretty print large data structures.

\item Make sure all your Erlang programs are in the same directory.
If you code is in different directories you might run into problems with code paths.

\item To compile all program in a directory give the shell command:
\verb+erlc *.erl+ (OS-X and Linux only, not windows).
\end{itemize}


\section{The test program}

\VerbatimInput[frame=single]{problems1.erl}


\end{document}
