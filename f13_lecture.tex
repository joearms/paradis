\documentclass[10pt]{article}

\usepackage{fancyvrb}
%% \usepackage{hyperref}

\title{Parallel and Distributed Programs\\
Lecture F13\\
25 February, 2014
}
\author{Joe Armstrong}

\fvset{frame=single,numbers=left,numbersep=3pt}

\begin{document}

\maketitle

\tableofcontents

\section{Introduction}

In this lecture we will put together some of the things we learned in
earlier lectures.

The focus of this lecture is to see how to make a bit-torrent like
system with a client, server and tracker. Why are we studying bit
torrent? The answer is easy. Bit torrent carries more traffic than any
other protocol on the internet. It is the primary protocol used for
content distribution. It's also a very good example of a parallel and
distributed program.

By the end of the course you should be able to design and program a
bit torrent like system and adapt it to your needs.

In earlier lectures we have covered:

\begin{itemize}
\item Creating a digest.
\item Making a UDP client and server.
\item Sending Erlang terms to UDP servers.
\end{itemize}

This is actually all we need to make a bit torrent like system, all we have
to do is put the pieces together.

\section{Bit Torrent Architecture}

Bit torrent consists of a number of different programs that {\sl collaborate} to
form a content distribution network. We need the following programs:

\begin{description}
  
\item \verb+make_torrent <file>+\\
Creates a torrent file. The torrent file contains
a digest\\
\verb+{TotLength, BlockSize, [Md5(Blocks)]}+

\item \verb+start_tracker <port>+\\
Starts a tracker listening for UDP messages on Port.

\item \verb+start_client <torrent_file> <tracker_IP> <tracker_port>+\\
Starts a client on your local machine. \verb+<torrent_file>+ is the 
name of the torrent. \verb+tracker_ip+ and \verb+tracker_port+
are the identity of the tracker. How the three arguments to
\verb+start_client+ are known is not part of the system. These
are usually obtained by an on-out-band mechanism.

\end{description}

\section{The Tracker and Client-Tracker Protocol}

The tracker maintains a state variable which keeps track of which
hosts are interested in which files.  The state is represented as a
tuple of the form \verb+[{SHA,[{IP,Port,Tadd}]}]+. \verb+SHA+ is the
SHA1 checksum of the file.  \verb+IP,Port+ is the IP address and port
number of a machine that wants to exchange the file. \verb+Tadd+ is
the time when the entry was added to the list. After 10 minutes the
entry should be automatically removed from the list.

The tracker can be sent one message: 

\begin{description}
\item \verb+{add_me,SHA,IP,Port}+\\
This mean the host \verb+IP+, \verb+Port+ is interested in the file
with hash \verb+SHA+. This should be sent once every ten minutes.
At \verb+{IP,Port,time_now()}+ tuple is added to the list of
hosts associated with the hash \verb+SHA+.

The server responds with a UDP packet to \verb+{IP,Port}+
containing a list of \verb+Port,IP+ machines what are interested in the 
file.

\end{description}

\section{The Client - Tracker protocol}

In normal operation, a client that wants to download a file does the
following:

\begin{itemize}
\item Obtains a torrent from somewhere. This is usually
  found on a web site, sent in an email, etc. How this 
is distributed is not part of the bit torrent protocol.

\item Finds the IP address and port number of a tracker.
  Again how this is done is not part of the bit torrent protocol.

\item Opens a UDP listening port at (\verb+IP+, \verb+Port+) and sends a\\
  \verb+{add_me, SHA, IP, Port}+ message to the tracker.

\item Waits for a response from the tracker. The tracker will hopefully
send back a list of machines.

\item Contacts the client and starts the file exchange protocol.
\end{itemize}

\section{The Client - Client Protocol}

Once a client has a list of \verb+IP+, \verb+Port+ pairs associated with a
hash \verb+H+ it can get get to work.

There are only a few messages in the Client-Client Protocol:

\begin{description}
\item \verb+{ihave1,SHA1,B1}+\\
\verb+B1+ is a bitmap of the blocks that I have.
\verb+SHA1+ is a SHA1 checksum of the torrent file.
the units of the bitmap are in MB blocks.

Since we are using UDP we won't want to send very large packets.
Assume we want to send a 1GB file. This is 1000 blocks of 1MB
which can be represented in a bitmap of 125 bytes. 
1GB is 1 million 1K Blocks which could be represented in a
125KByte bitmap - but this is too big for a UDP packet, so we use MB units
in the bitmap.

Our strategy is to first agree on which 1 MB blocks we want to
transfer then on which KB blocks we want to exchange.

\item \verb+{ihave2,SHA1,K,B}+\\
This is similar to the previous message but now it is talking about
block \verb+K+. Block \verb+K+ is the \verb+K+'th MB block in the file.
\verb+B+ is a bit map of the KB data that I have in this block.

\item \verb+{data,SHA1,K,Min,Max,Data}+\\
Used by a peer to send data to a peer.  
\verb+K+ is the MB block number
\verb+Min+ and \verb+Max+ are the \verb+Min+ and \verb+Max+ 
are the Kbyte numbers of the data within the block.

\item \verb+{iwant,SHA1,K,B}+\\
  Means I want any of the blocks in MB block \verb+K+ that have a
  1 bit set in the bitmap B.

\end{description}

If we have several simultaneous connected peers keeping track of all the
requests and of who has and who wants what is non-trivial exercise
in concurrent programming.

Note that the entire protocol assumes that clients will share data on
collaborative basis, so some measure of trust is involved. If I send
you data, I will want you to send me data in return. If I send you a
lot of data and you send me little data, or data that is corrupt, then
I will stop talking to you and stop sending you data. I might even
send you a large number of small blocks of corrupt data and
blacklist your IP.

\section{Dimensioning}

In problems like this we always have to perform some numerical estimates
of the packet sizes involved.

\begin{enumerate}

\item Assume a max file size of 16GB. This is 16K x 1MByte blocks
  so the size of the MB block bitmap is 2KBytes.

  Note: Less than 576 is absolutely safest.
  A one GB file has a 128 Byte bitmap and will be very safe.

\item The size of the KB bitmap within a MB block is
  128 bytes.

\item The size of the digest is 16 bytes per block (for MD5)
so a 1GB file will have 1024 blocks x 16 bytes = 16KB. 

\end{enumerate}

\section{Discussion}

\begin{itemize}

\item What are the block sizes in the torrent?
\item What is the checksum algorithm (SHA1, MD5, merkle-hash-trees)?
\item The real protocol has things like \verb+choke+ \verb+unchoke+
etc. Why are these needed?
\item Can we ``tune'' the size of the UDP blocks. We'd like to monitor
  how well things were going and send larger blocks if we could.
\item Security. Should we encrypt the P2P exchanges, or is the
digest checksum provide sufficient security?
\item How can we detect and ban rogue peers that try to disrupt the
system? 
  
\end{itemize}

\section{References}

\begin{itemize}

\item  https://wiki.theory.org/BitTorrentSpecification

\item 
  ``Rarest First and Choke Algorithms are Enough''
http://arxiv.org/pdf/cs/0609026.pdf

\end{itemize}

\end{document}


